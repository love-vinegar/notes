\documentclass{article}
\usepackage[utf8]{inputenc}

\title{AG1}
\author{David Ployer}
\date{September 2022}

\begin{document}

\maketitle
\section{Přednáška 1}
Primitivni funkce 
	Definice 
		Nechť funkce f je definovaná na intervalu (a, b), kde -\infnt <= a < b <= \infnt . Funkci F splňnující podmínku 
		F'(x) = f(x)

Neurcitý integral 
	Necht F je primitivní funcí k funkci f na intervalu (a, b) Pak G je primitivní fuknicí k funkci f na intervalu (a, b) právě tehdz kdyz existuje konstanta C z R takova ze 
	G(x) = F(x) + C pro každe x z (a, b)
	definice 
		necht k funkci f existuje primitivni funkce na intervalu a, b Množinu všch primitivních funkcí k funkci f na (a, b) nazáváme neurčitým integralem funcke f na intervalu (a, b) značíme jej (znak integralu) f
	defivace a intefrace jsou inverzní.
	//vložit tabulku integrací a derivací

Vlastnosti neurciteho integralu
	nechť F, G je primitivni fukkci f, g na intervalu (a, b) a necht \alpha z R pak 
		F+G je primigivni fukci k f+g na (a, b)
		\alpha F je primitivní funkcí k funkc9 \alpha f na (a, b)
	
Existence primitivní funkce
	nechť funkce f je spojitá na intervalu (a, b). pak funkce f má na tomto intervalu primitivní funkci.

Integrace per partes 
	nechť funkce f je diferencovatelná v intervalu (a, b) a G je primitivní funkce k funcki g na intervalu (a, b) a konečně nechť existuje primitivní funkce k funkci f'G potom existuje primitivní funkce k fuknci fg a plati
$$\integrace fg = fG - \integrace f'G$$
	davat pozor na to jakou funkci dam za f a jakou za g

Substituce v neurčítém integrálu

\section{Přednáška 2}

\end{document}
