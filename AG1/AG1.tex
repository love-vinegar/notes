\documentclass{article}
\usepackage[utf8]{inputenc}

\title{AG1}
\author{David Ployer}
\date{September 2022}

\begin{document}

\maketitle
\section{Přednáška 1}
Neorientovaný graf je uspořádaná dvojice (V,E), kde 
	V je neprázdná konečná množina vrcholů
	E je množina hran
Hrana je dvojprvková podmnožiná V, značíme {u, v}

nechť e = {u, v}  je hrana v grafu G. Pak řekneme, že 
	vrcholy u a v jsou koncové vrcholy hrany e
	u je sousedem v v G
	u i v jsou inicidentní s hranou e
Sled délky k v gragu G je sekvence v\_0, e\_1, v\_1 .. e\_kv\_k taková  že ei = {vi-1, vi} a ei \in E(G) pro všechna i
cesta v grafu G je sled ve kterem se neopakuji vrcholy

BFS
	Algoritmus BFS(G, S) se vždz zastaví.
	
	věta o správnosti algoritmu BFS
\begin{itemize}
	\item Po skončení BFS(G, s) jsou uzavřené právě ty vrcholy,
	do kterých vede cesta ze startu s a ostatní vrcholy
zůstanou nenalezené.

	\item Pro všechny uzavřené vrcholy v platí D(v) = d(s, v) =
	délka nejkratší cesty ze startu s do vrcholu v.

	\item Pro všechny uzavřené vrcholy v platí P(v) = w, kde w
	je předchůdce v na nějaké nejkratší cestě ze startu s do
vrcholu v.
\end{itemize}

úplný graf K\_{n}
nechť n \geq 1
Úplný graf na n vrcholech K\_n je grav (V, (v2)) kde \abs{V} = n.
neboli kazdy s kazdym

uplny bipartitni graf
dve zkupiny mezi kterýma kazdy s kazdym ale ve skupine nikdo s nikým

cesta P\_m
nechť m \geq 0 
cesta délky m (s m hranami) P\_m je graf 
({0, ... , m}, {{i, i + 1} | i \in {0, ... , m − 1}})


kružnice 
nechť n \geq 3
kružnice délky n (s n vrcholy) C\_n je graf kterej je proste do kruhu

princip sudosti 
Pro každý graf G = (V, E) platí
$$\sum_{v \in V}deg\_{G}(v) = 2\abs{E} $$

\end{document}
